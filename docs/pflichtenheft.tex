\documentclass[12pt,titlepage]{article}

\usepackage{ngerman,xunicode}

\begin{document}

\title{Miniprojekt: \\ Minimax-Maschine}
\author{Clemens Pollak, Robin Thrift, Max Boll}
\date{2014}
\maketitle


\section{Einleitung} 
Unser gew{\"a}hltes Thema befasst sich mit der Minimax-Maschine und mit der Realisierung von Algorithmen auf ihr. Dieses Thema soll im Rahmen des Hardware Praktikums bearbeitet werden und ist uns aus der Veranstaltung "'Grundlagen der Rechnerarchitektur"' bereits grundlegend bekannt.\\ Um unsere Vorbereitung auf dieses Projekt dokumentieren und strukturieren zu k{\"o}nnen, erstellen wir ein Pflichtenheft. Dieses Pflichtenheft wird nur unsere Vorbereitung beinhalten und es wird au{\ss}erdem eine weitere Dokumentation der Ergebnissen geben.

\section{Aufgabenstellung}
Nach unserem Verst{\"a}ndniss ist das Ziel dieser Aufgabenstellung einen Algorithmus, welcher auf der Minimax-Maschine zu implementieren ist und eine sogenannte Paketanalyse betreibt. Dieser "'Paketanalyse"'-Algorithmus befasst sich mit dem Datenpaketen, welche im Speicher der Minimax-Maschine abgelegt sind.\\ Ein Datenpaket beginnt immer mit der Folge "'1110"' und besteht aus einem
Kopf mit einer L{\"a}nge von 80 Bits und einem Datenteil mit variabler Länge. Der Kopf enth{\"a}lt die Kanalnummer zwischen der 32. und 47. Bitstelle. Zu einem Kanal können ein oder mehrere Pakete gehören, welche die selbe Kanalnummer haben.
Die Länge des kompletten Paketfeldes im Speicher wird als bekannt vorausgesetzt. Deswegen werden die einzelnen L{\"a}ngen der Pakete in ein entsprechendes Register vorgeladen.\\
Nun soll der Algorithmus eine Datentabelle anlegen, welche sich au{\ss}erhalb der Speicherfelder der einzelnen Pakete befindet. Diese Datentabelle soll den Kanalnummern die L{\"a}ngen der jeweiligen Datenteile aller Pakete zuordenen. Diese Aufgabenstellung soll mit dem Minimax-Simulator simuliert und getestet werden. Die Maschine kann durch vorgegebene Bauteile erweitert werden, was sich aber auf die Bewertung auswirkt.


\section{Ist-Analyse der Basis-Maschine}

Die Minimax-Maschine ist ein minimales Rechensystem welches aus einfachen Registern (Basis: `ACCU`, `PC`, `IR`, `MDR`, `MAR`;
weitere können hinzugeüfgt werden), einer arithmetisch-logischen Einheit (ALU) und einem Hauptspeicher (HS) aufgebaut
ist und durch ein Mikropgramm gesteuert wird. Dabei sind die möglichen Operationen auf die in der ALU implementieren
Operationen beschränkt (Basis: 'ADD', 'SUB.B', 'TRANS.A', 'TRANS.B'). Die ALU kannn mit weiteren Operationen,
wie z. B. dem bitweisen UND, ergänzt werden.


Um eine Operation auszuführen müssen über die Multiplexer `ALUSel.A` und `AluSel.B` zwei Operanden ausgewählt werden
und der ALU muss über die `ALU Ctrl`-Leitung der Code für die Operation übergben werden. Das Ergebnis der Operation kann
entweder in einem Register oder im HS (Adresse im Register `MAR`) gespeichert werden. Zusätzlich können s. g. Flags 
gesetzt werden, welche zurück zur Control Unit (CU) geleitet werden um z. B. bedingte Sprünge auszuführen.


Die uns vorliegende Minimax-Maschine arbeitet mit 32-Bit und speichert Werte mit 32-Bit in den Registern und im HS.
Alle ALU-Operationen werden folglich alle mit 32-Bit ausgeführt. Dies stellt sich jedoch für unser Aufgabe als
Hindernis, da wir die Daten bitweise untersuchen müssen, Daten aus dem HS und den Registern jedoch nur als
32-Bit Zahlen auslesen können und nicht als einzelne Bits.


\end{document}
