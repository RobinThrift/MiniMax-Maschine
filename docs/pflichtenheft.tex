\documentclass[12pt,titlepage]{article}

\usepackage{ngerman}

\begin{document}

\title{Miniprojekt: \\ Minimax-Maschine}
\author{Clemens Pollak, Robin Thrift, Max Boll}
\date{23.10.14}
\maketitle


\section{Einleitung} 
Unser gew{\"a}hltes Thema befasst sich mit der Minimax-Maschine und mit der Realisierung von Algorithmen auf ihr. Dieses Thema soll im Rahmen des Hardware Praktikums bearbeitet werden und ist uns aus der Veranstaltung "'Grundlagen der Rechnerarchitektur"' bereits grundlegend bekannt.\\ Um unsere Vorbereitung auf dieses Projekt dokumentieren und strukturieren zu k{\"o}nnen, erstellen wir ein Pflichtenheft. Dieses Pflichtenheft wird nur unsere Vorbereitung beinhalten und es wird au{\ss}erdem eine weitere Dokumentation der Ergebnissen geben.

\section{Aufgabenstellung}
Nach unserem Verst{\"a}ndniss ist das Ziel dieser Aufgabenstellung einen Algorithmus, welcher auf der Minimax-Maschine zu implementieren ist und eine sogenannte Paketanalyse betreibt. Dieser "'Paketanalyse"'-Algorithmus befasst sich mit dem Datenpaketen, welche im Speicher der Minimax-Maschine abgelegt sind.\\ Ein Datenpaket beginnt immer mit der Folge "'1110"' und besteht aus einem
Kopf mit einer L{\"a}nge von 80 Bits und einem Datenteil mit variabler Länge. Der Kopf enth{\"a}lt die Kanalnummer zwischen der 32. und 47. Bitstelle. Zu einem Kanal können ein oder mehrere Pakete gehören, welche die selbe Kanalnummer haben.
Die Länge des kompletten Paketfeldes im Speicher wird als bekannt vorausgesetzt. Deswegen werden die einzelnen L{\"a}ngen der Pakete in ein entsprechendes Register vorgeladen.\\
Nun soll der Algorithmus eine Datentabelle anlegen, welche sich au{\ss}erhalb der Speicherfelder der einzelnen Pakete befindet. Diese Datentabelle soll den Kanalnummern die L{\"a}ngen der jeweiligen Datenteile aller Pakete zuordenen. Diese Aufgabenstellung soll mit dem Minimax-Simulator simuliert und getestet werden. Die Maschine kann durch vorgegebene Bauteile erweitert werden, was sich aber auf die Bewertung auswirkt.
\\
\\
\\

\section{Ist-Analyse der Basis-Maschine}


\section{Angestrebte Projektergebnisse}
\begin{enumerate}
\item Die Tabelle, in der die einzelnen Kanalnummern und Paketlängen enthalten sind. (Als exportierte Speicherdatei)
\item Eine Dokumentation, welche unserer Implementierung im Detail beschreibt und m{\"o}glicherweise schwierige Stellen verst{\"a}ndlich erl{\"a}utert.
\item Einen exportierten Schaltplan der erweiterten Minimax-Maschine.
\end{enumerate}


\end{document}
